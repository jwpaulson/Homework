\documentclass[11pt]{article}
\usepackage{amsmath}
\usepackage{amssymb}
\usepackage{epstopdf}
\usepackage{fullpage}
\usepackage[parfill]{parskip}
\usepackage{graphicx}

\pdfpagewidth 8.5in
\pdfpageheight 11in

\title{AMATH 500/STAT 591A: Homework 1}
\author{Joshua Paulson}
\date{14 - October - 2010}

\begin{document}
\maketitle{\textbf{}}\\
Problem Number 1) a.\\

\indent Considering the equations:\\
$\mu_{p} = w_{0}\mu_{1} + (1-w_{0})\mu_{2}$\\
$\sigma^{2}_{p} = w^{2}_{0}\sigma^{2}_{1} + 2w_{0}(1 - w_{0})\rho*\sigma_{1}\sigma_{2} + (1 - w_{0})^{2} \sigma^{2}_{2}$\\
$\sigma_{p} = \sqrt{w^{2}_{0}\sigma^{2}_{1} + 2w_{0}(1 - w_{0})\rho*\sigma_{1}\sigma_{2} + (1 - w_{0})^{2} \sigma^{2}_{2}}$\\
\\
if $\rho = +1$, then assets one and two are perfectly correlated thus, $\sigma_{1}\sigma_{2} = \sigma_{1 2}$ and\\
$\sigma_{p} = w_{0}\sigma_{1} + (1 - w_{0})\sigma_{2}$, since complete square\\
Then,\\
$\sigma_{p} = w_{0}\sigma_{1} + \sigma_{2} - w_{0}\sigma_{2}$\\
$\sigma_{p} - \sigma_{2} = w_{0}\sigma_{1} - w_{0}\sigma_{2}$\\
$\sigma_{p} - \sigma_{2} = w_{0}(\sigma_{1} - \sigma_{2})$\\
$\frac{\sigma_{p} - \sigma_{2}}{(\sigma_{1} - \sigma_{2})} = w_{0}$\\

$\mu_{p} = w_{0}\mu_{1} + (1-w_{0})\mu_{2}$\\
$\mu_{p} = \frac{\sigma_{p} - \sigma_{2}}{(\sigma_{1} - \sigma_{2})}\mu_{1} + (1-\frac{\sigma_{p} - \sigma_{2}}{(\sigma_{1} - \sigma_{2})})\mu_{2}$\\
$\mu_{p} = \frac{\sigma_{p} - \sigma_{2}}{(\sigma_{1} - \sigma_{2})}\mu_{1} + \mu_{2} - \mu_{2}\frac{\sigma_{p} - \sigma_{2}}{(\sigma_{1} - \sigma_{2})}$\\
$\mu_{p} = \mu_{2} + (\mu_{1} - \mu_{2})\frac{\sigma_{p} - \sigma_{2}}{(\sigma_{1} - \sigma_{2})}$\\
If we set $a = \mu_{2}$, and $b = (\mu_{1} - \mu_{2})$, and consider that $\sigma_{1}$ and $\sigma_{2}$ are fixed numbers,\\
then we can see this equation fits the form of $\mu_{p} = a + b\sigma_{p}$ which is a linear relationship.\\

Problem Number 1) b.\\

$\mu_{p} = w_{0}\mu_{1} + (1-w_{0})\mu_{2}$\\
$\sigma^{2}_{p} = w^{2}_{0}\sigma^{2}_{1} + 2w_{0}(1 - w_{0})\rho*\sigma_{1}\sigma_{2} + (1 - w_{0})^{2} \sigma^{2}_{2}$\\
$\sigma_{p} = \sqrt{w^{2}_{0}\sigma^{2}_{1} + 2w_{0}(1 - w_{0})\rho*\sigma_{1}\sigma_{2} + (1 - w_{0})^{2} \sigma^{2}_{2}}$\\
\\
if $\rho = -1$, then assets one and two have perfect negative correlation thus, $-\sigma_{1}\sigma_{2} = \sigma_{1 2}$ and\\
$\sigma_{p} =$ absolute value of $(w_{0}\sigma_{1} - (1 - w_{0})\sigma_{2})$\\
Due to the negative, we see two different equations\\
The first:\\
$\sigma_{p} = w_{0}\sigma_{1} - \sigma_{2} + w_{0}\sigma_{2}$\\
$\sigma_{p} + \sigma_{2} = w_{0}\sigma_{1} + w_{0}\sigma_{2}$\\
$\sigma_{p} + \sigma_{2} = w_{0}(\sigma_{1} + \sigma_{2})$\\
$\frac{\sigma_{p} + \sigma_{2}}{(\sigma_{1} + \sigma_{2})} = w_{0}$\\

$\mu_{p} = w_{0}\mu_{1} + (1-w_{0})\mu_{2}$\\
$\mu_{p} = \frac{\sigma_{p} + \sigma_{2}}{(\sigma_{1} + \sigma_{2})}\mu_{1} + (1-\frac{\sigma_{p} + \sigma_{2}}{(\sigma_{1} + \sigma_{2})})\mu_{2}$\\
$\mu_{p} = \frac{\sigma_{p} + \sigma_{2}}{(\sigma_{1} + \sigma_{2})}\mu_{1} + \mu_{2} - \mu_{2}\frac{\sigma_{p} + \sigma_{2}}{(\sigma_{1} + \sigma_{2})}$\\
$\mu_{p} = \mu_{2} + (\mu_{1} - \mu_{2})\frac{\sigma_{p} + \sigma_{2}}{(\sigma_{1} + \sigma_{2})}$\\
If we set $a = \mu_{2}$, and $b = (\mu_{1} - \mu_{2})$, and consider that $\sigma_{1}$ and $\sigma_{2}$ are fixed numbers,\\
then we can see this equation fits the form of $\mu_{p} = a + b\sigma_{p}$ which is a linear relationship.\\
\\
The second:\\
$\sigma_{p} = -w_{0}\sigma_{1} + \sigma_{2} - w_{0}\sigma_{2}$\\
$\sigma_{p} - \sigma_{2} = -w_{0}\sigma_{1} - w_{0}\sigma_{2}$\\
$\sigma_{p} - \sigma_{2} = w_{0}(-\sigma_{1} - \sigma_{2})$\\
$\frac{\sigma_{p} - \sigma_{2}}{(-\sigma_{1} - \sigma_{2})} = w_{0}$\\

$\mu_{p} = w_{0}\mu_{1} + (1-w_{0})\mu_{2}$\\
$\mu_{p} = \frac{\sigma_{p} - \sigma_{2}}{(-\sigma_{1} - \sigma_{2})}\mu_{1} + (1-\frac{\sigma_{p} - \sigma_{2}}{(-\sigma_{1} - \sigma_{2})})\mu_{2}$\\
$\mu_{p} = \frac{\sigma_{p} - \sigma_{2}}{(-\sigma_{1} - \sigma_{2})}\mu_{1} + \mu_{2} - \mu_{2}\frac{\sigma_{p} - \sigma_{2}}{(-\sigma_{1} - \sigma_{2})}$\\
$\mu_{p} = \mu_{2} - (\mu_{1} - \mu_{2})\frac{\sigma_{p} - \sigma_{2}}{(\sigma_{1} + \sigma_{2})}$, (pulling the negative out)\\
If we set $a = \mu_{2}$, and $b = (\mu_{1} - \mu_{2})$, and consider that $\sigma_{1}$ and $\sigma_{2}$ are fixed numbers,\\
then we can see this equation fits the form of $\mu_{p} = a - b\sigma_{p}$ which is a negative linear relationship.\\
\\

Problem Number 2) a.\\

Since the assets are uncorrelated, thier covariances will equal zero\\
Further, $\mu_{p} = E(\sum_{i=1}^{n} w_{i}r_{i}) = \sum_{i=1}^{n} w_{i}r_{i} = \sum_{i=1}^{n} w_{i}E(r_{i}) = \mu$\\
Thus $\mu$ is a fixed constant and the $\bar{r}-\sigma$ diagram will simply be a horizontal line at $\mu$.\\
The efficient set will be the point furtherest to the left on the line, which has the minimum variance for the level $\mu$. This is also the minimum-variance point.\\

Problem Number 2) b.\\

To find the minimum-variance point we use the Lagrangian method to minimize the equation subject to the constraint. Thus,\\
minimize $\frac{1}{2}\sum_{i=1}^{n}w_{i}^{2}\sigma_{i}^{2}$\\
subject to $\sum_{i=1}^{n}w_{i} = 1$\\

L = $\frac{1}{2}\sum_{i=1}^{n}w_{i}^{2}\sigma_{i}^{2} - \lambda(\sum_{i=1}^{n}w_{i} - 1)$\\
$\frac{\delta L}{\delta w_{i}} = w_{i}\sigma_{i}^{2} - \lambda = 0$ \hspace{5mm} and \hspace{5mm} $\frac{\delta L}{\delta \lambda} = \sum_{i=1}^{n}w_{i} - 1 = 0$\\
Thus, $w_{i} = \frac{\lambda}{\sigma_{i}^{2}}$ and $\sum_{i=1}^{n} \frac{\lambda}{\sigma_{i}^{2}} = 1$ \hspace{3mm} $\Rightarrow$ \hspace{3mm} $\lambda = \bar{\sigma}^{2}$ \hspace{3mm} so $w_{i} = \frac{\bar{\sigma}^{2}}{\sigma_{i}^{2}}$\\

Now considering:\\
$\sigma_{MV}^{2} = \sum_{i=1}^{n}w_{i}^{2}\sigma_{i}^{2}$\\
$\sigma_{MV}^{2} = \sum_{i=1}^{n} (\frac{\lambda}{\sigma_{i}^{2}})^{2}\sigma_{i}^{2}$\\
$\sigma_{MV}^{2} = \sum_{i=1}^{n}\frac{\lambda^{2}}{\sigma_{i}^{4}}\sigma_{i}^{2}$\\
$\sigma_{MV}^{2} = \sum_{i=1}^{n}\frac{\lambda^{2}}{\sigma_{i}^{2}}$\\
$\sigma_{MV}^{2} = \lambda^{2} \sum_{i=1}^{n}\frac{1}{\sigma_{i}^{2}}$\\

Now replacing $\lambda$\\
$\sigma_{MV}^{2} = \frac{(\bar{\sigma}^{2})^{2}}{\bar{\sigma}^{2}}$\\
$\sigma_{MV}^{2} = \bar{\sigma}^{2}$\\




Problem Number 3) a.\\

Note $w_{1} = w_{3}$\\
Using the equation:\\
$w_{MV} = \frac{\Omega^{-1} 1}{1^{'} \Omega^{-1} 1}$\\

\[
\Omega = \left( \begin{array}{ccc}
2 & 1 & 0 \\
1 & 2 & 1 \\
0 & 1 & 2 \end{array} \right)
\\
\hspace{3mm} Thus, \hspace{3mm}
\Omega^{-1} = \left( \begin{array}{ccc}
\frac{3}{4} & \frac{-1}{2} & \frac{1}{4} \\
\frac{-1}{2} & 1 & \frac{-1}{2} \\
\frac{1}{4} & \frac{-1}{2} & \frac{3}{4} \end{array} \right)
\\
\]

so\\

\[
w_{MV} =
\frac{
 \left( \begin{array}{ccc}
\frac{3}{4} & \frac{-1}{2} & \frac{1}{4} \\
\frac{-1}{2} & 1 & \frac{-1}{2} \\
\frac{1}{4} & \frac{-1}{2} & \frac{3}{4} \end{array} \right)
\\
\left( \begin{array}{ccc}
1 \\
1 \\
1 \end{array} \right)
}
{
\left( \begin{array}{ccc}
1 & 1 & 1 \end{array} \right)
\\
\left( \begin{array}{ccc}
\frac{3}{4} & \frac{-1}{2} & \frac{1}{4} \\
\frac{-1}{2} & 1 & \frac{-1}{2} \\
\frac{1}{4} & \frac{-1}{2} & \frac{3}{4} \end{array} \right)
\\
\left( \begin{array}{ccc}
1 \\
1 \\
1 \end{array} \right)
\\
}
\hspace{3mm}
=
\hspace{3mm}
w_{MV} =
\frac{
\left( \begin{array}{ccc}
\frac{1}{2} \\
0 \\
\frac{1}{2} \end{array} \right)
}
{
\left( \begin{array}{c}
1 \end{array} \right)
}
\]

Thus, the minimum-variance portfolio is $w_{1} = \frac{1}{2}, w_{2} = 0, w_{3} = \frac{1}{2}$ or $w = (0.5, 0, 0.5)$\\

Problem Number 3) c.\\
$w_{T} = \frac{\Omega^{-1}\mu_{e}}{1^{'}\Omega^{-1}\mu_{e}}$\\

Since $r_{f} = 0.2$ and $\bar{r} = [\begin{smallmatrix} 0.4\\ 0.8\\ 0.8 \end{smallmatrix}]$, then $\mu_{e} = [\begin{smallmatrix} 0.2\\ 0.6\\ 0.6 \end{smallmatrix}]$\\

so\\

\[
w_{T} =
\frac{
\left( \begin{array}{ccc}
\frac{3}{4} & \frac{-1}{2} & \frac{1}{4} \\
\frac{-1}{2} & 1 & \frac{-1}{2} \\
\frac{1}{4} & \frac{-1}{2} & \frac{3}{4} \end{array} \right)
\\
\left( \begin{array}{ccc}
0.2 \\
0.6 \\
0.6 \end{array} \right)
}
{
\left( \begin{array}{ccc}
1 & 1 & 1 \end{array} \right)
\\
\left( \begin{array}{ccc}
\frac{3}{4} & \frac{-1}{2} & \frac{1}{4} \\
\frac{-1}{2} & 1 & \frac{-1}{2} \\
\frac{1}{4} & \frac{-1}{2} & \frac{3}{4} \end{array} \right)
\\
\left( \begin{array}{ccc}
0.2 \\
0.6 \\
0.6 \end{array} \right)
\\
}
\hspace{3mm}
=
\hspace{3mm}
w_{T} =
\frac{
\left( \begin{array}{ccc}
0 \\
0.2 \\
0.2 \end{array} \right)
}
{
\left( \begin{array}{c}
0.4 \end{array} \right)
}
\]

Thus, the efficient portfolio is $w_{1} = 0, w_{2} = \frac{1}{2}, w_{3} = \frac{1}{2}$ or $w = (0, 0.5, 0.5)$\\


Problem Number 4)\\

We want to maximize $U(w) = w^{'}\mu - \frac{1}{2} \lambda w^{'}\Omega w$, subject to $w^{'}1 = 1$\\
Thus, we set up the Lagrangian:\\
$\mu - \lambda \Omega w - \gamma 1 = 0$ and $w^{'}1$\\
so, $w = \lambda^{-1} \Omega^{-1} (\mu - \gamma 1)$\\
Solving for $\gamma$:\\
$w^{'} 1 = \lambda^{-1} 1^{'} \Omega^{-1} \mu - \lambda^{-1} \gamma 1^{'} \Omega^{-1} 1 = 1$\\
if we set $a = 1^{'} \Omega^{-1} \mu$, $c = 1^{'} \Omega^{-1} 1$\\
then, $1 = \frac{a}{\lambda} - \frac{\gamma c}{\lambda} \Rightarrow 1 = \frac{a - \gamma c}{\lambda} \Rightarrow \lambda = a - \gamma c \Rightarrow \gamma c = a - \lambda \Rightarrow \gamma = \frac{a}{c} - \frac{\lambda}{c}$\\
Plugging this result into the equation for $w = \lambda^{-1} \Omega^{-1} (\mu - \gamma 1)$:\\

$w = \lambda^{-1} \Omega^{-1} \mu - \lambda^{-1} \Omega^{-1} \gamma 1)$\\
$w = \lambda^{-1} \Omega^{-1} \mu - \lambda^{-1} \Omega^{-1} (\frac{a}{c} - \frac{\lambda}{c}) 1)$\\
$w = \lambda^{-1} \Omega^{-1} \mu - (\frac{\lambda^{-1} \Omega^{-1} a}{c} - \frac{\lambda^{-1} \Omega^{-1} \lambda}{c}) 1)$\\
$w = \lambda^{-1} \Omega^{-1} \mu - (\frac{\lambda^{-1} \Omega^{-1} 1^{'} \Omega^{-1} \mu}{1^{'} \Omega^{-1} 1} - \frac{\lambda^{-1} \Omega^{-1} \lambda}{1^{'} \Omega^{-1} 1}) 1)$\\
$w = \lambda^{-1} \Omega^{-1} \mu - (\lambda^{-1} 1^{'} \Omega^{-1} \mu - \lambda^{-1} \lambda) \frac{\Omega^{-1} 1}{{1^{'} \Omega^{-1} 1}}$\\
$w = \lambda^{-1} \Omega^{-1} \mu - (\lambda^{-1} 1^{'} \Omega^{-1} \mu - 1) W_{MV}$\\
Since $1^{'}*W_{G} = 1$, we can multiply it into the equation\\
$w = (\lambda^{-1} 1^{'} \Omega^{-1} \mu) W_{G} - (\lambda^{-1} 1^{'} \Omega^{-1} \mu - 1) W_{MV}$\\
$w = (\lambda^{-1} 1^{'} \Omega^{-1} \mu) W_{G} + (-\lambda^{-1} 1^{'} \Omega^{-1} \mu + 1) W_{MV}$\\
$w = (\lambda^{-1} 1^{'} \Omega^{-1} \mu) W_{G} + (1 - \lambda^{-1} 1^{'} \Omega^{-1} \mu) W_{MV}$\\
Thus we have shown the Two-Fund Separation Theorem\\

Problem Number 5)\\
$\Omega = [\begin{smallmatrix} \sigma_{11} & \sigma_{12}\\ \sigma_{21} & \sigma_{22} \end{smallmatrix}]$, First we need to compute $\Omega^{-1}$\\
$\Omega^{-1} = [\begin{smallmatrix} \sigma_{22} & -\sigma_{12}\\ -\sigma_{21} & \sigma_{11} \end{smallmatrix}]* \frac{1}{\sigma_{11}\sigma_{22} - \sigma_{12}\sigma_{21}}$\\
$\mu = [\begin{smallmatrix} \mu_{1}\\ \mu_{2} \end{smallmatrix}]$\\

$w = \lambda^{-1} \Omega^{-1} \mu$\\
$w = \frac{1}{\lambda} \frac{1}{\sigma_{11}\sigma_{22} - \sigma_{12}\sigma_{21}} [\begin{smallmatrix} \sigma_{22} \mu_{1} - \sigma_{12} \mu_{2} \\ \sigma_{21} \mu_{1} + \sigma_{11} \mu_{2} \end{smallmatrix}]$\\
Thus we can get each w individually, then take the derivatives\\
$w_{1} = \frac{1}{\lambda} \frac{\sigma_{22} \mu_{1} - \sigma_{12} \mu_{2}}{\sigma_{11}\sigma_{22} - \sigma_{12}\sigma_{21}}$\\
$\frac{\delta w_{1}}{\delta \mu_{1}} = \frac{1}{\lambda} \frac{\sigma_{22}}{\sigma_{11}\sigma_{22} - \sigma_{12}\sigma_{21}}$\\

since\\
$\sigma_{12} = \sqrt{\sigma_{11}} \sqrt{\sigma_{22}} \rho$\\
$\sigma_{21} = \sqrt{\sigma_{11}} \sqrt{\sigma_{22}} \rho$\\
$\sigma_{12} \sigma_{21} = \sigma_{11} \sigma_{22} \rho^{2}$\\

$\frac{\delta w_{1}}{\delta \mu_{1}} = \frac{1}{\lambda} \frac{\sigma_{22}}{\sigma_{11}\sigma_{22} - \sigma_{11}\sigma_{22} \rho^{2}}$\\
$\frac{\delta w_{1}}{\delta \mu_{1}} = \frac{1}{\lambda} \frac{\sigma_{22}}{\sigma_{11}\sigma_{22} (1 - \rho^{2})}$\\
$\frac{\delta w_{1}}{\delta \mu_{1}} = \frac{1}{\lambda} \frac{1}{\sigma_{11} (1 - \rho^{2})}$\\

We can see that as $\rho$ approaches the absolute value of one, it will make the demoninator in this expression closer to zero, thus the overall expression will greatly increase. Thus we can see that a small change in $\mu$ will result in a large change in $w_{1}$. We will now look at the other derivates to confirm this.\\

Second:\\
$w_{1} = \frac{1}{\lambda} \frac{\sigma_{22} \mu_{1} - \sigma_{12} \mu_{2}}{\sigma_{11}\sigma_{22} - \sigma_{12}\sigma_{21}}$\\
$\frac{\delta w_{1}}{\delta \mu_{2}} = \frac{1}{\lambda} \frac{-\sigma_{12}}{\sigma_{11}\sigma_{22} - \sigma_{12}\sigma_{21}}$\\

since\\
$\sigma_{12} = \sqrt{\sigma_{11}} \sqrt{\sigma_{22}} \rho$\\
$\sigma_{21} = \sqrt{\sigma_{11}} \sqrt{\sigma_{22}} \rho$\\
$\sigma_{12} \sigma_{21} = \sigma_{11} \sigma_{22} \rho^{2}$\\

$\frac{\delta w_{1}}{\delta \mu_{2}} = \frac{1}{\lambda} \frac{-\sigma_{12}}{\sigma_{11}\sigma_{22} - \sigma_{11}\sigma_{22} \rho^{2}}$\\
$\frac{\delta w_{1}}{\delta \mu_{2}} = \frac{1}{\lambda} \frac{-\sigma_{12}}{\sigma_{11}\sigma_{22} (1 - \rho^{2})}$\\

Third:\\
$w_{2} = \frac{1}{\lambda} \frac{1}{(\sigma_{11}\sigma_{22} - \sigma_{12}\sigma_{21})(-\sigma_{21} \mu_{1} + \sigma_{11} \mu_{2})}$\\
$w_{2} = \frac{1}{\lambda} \frac{1}{-\sigma_{11}\sigma_{22}\sigma_{21}\mu_{1} + \sigma_{11}\sigma_{22}\sigma_{11}\mu_{2} + \sigma_{12}\sigma_{21}\sigma_{21}\mu_{1} - \sigma_{12}\sigma_{21}\sigma_{11}\mu_{2}}$\\
$\frac{\delta w_{2}}{\delta \mu_{1}} = \frac{1}{\lambda} \frac{1}{-\sigma_{11}\sigma_{22}\sigma_{21} + \sigma_{12}\sigma_{21}\sigma_{21}}$\\

since\\
$\sigma_{12} = \sqrt{\sigma_{11}} \sqrt{\sigma_{22}} \rho$\\
$\sigma_{21} = \sqrt{\sigma_{11}} \sqrt{\sigma_{22}} \rho$\\
$\sigma_{12} \sigma_{21} = \sigma_{11} \sigma_{22} \rho^{2}$\\

$\frac{\delta w_{2}}{\delta \mu_{1}} = \frac{1}{\lambda} \frac{1}{-\sigma_{11}\sigma_{22}\sigma_{21} + \sigma_{21}\sigma_{11} \sigma_{22} \rho^{2}}$\\
$\frac{\delta w_{2}}{\delta \mu_{1}} = \frac{1}{\lambda} \frac{1}{\sigma_{11}\sigma_{22}\sigma_{21} (-1 + \rho^{2})}$\\

Fourth:\\
$w_{2} = \frac{1}{\lambda} \frac{1}{(\sigma_{11}\sigma_{22} - \sigma_{12}\sigma_{21})(-\sigma_{21} \mu_{1} + \sigma_{11} \mu_{2})}$\\
$w_{2} = \frac{1}{\lambda} \frac{1}{-\sigma_{11}\sigma_{22}\sigma_{21}\mu_{1} + \sigma_{11}\sigma_{22}\sigma_{11}\mu_{2} + \sigma_{12}\sigma_{21}\sigma_{21}\mu_{1} - \sigma_{12}\sigma_{21}\sigma_{11}\mu_{2}}$\\
$\frac{\delta w_{2}}{\delta \mu_{2}} = \frac{1}{\lambda} \frac{1}{\sigma_{11}\sigma_{22}\sigma_{11} - \sigma_{12}\sigma_{21}\sigma_{11}}$\\

$\sigma_{12} = \sqrt{\sigma_{11}} \sqrt{\sigma_{22}} \rho$\\
$\sigma_{21} = \sqrt{\sigma_{11}} \sqrt{\sigma_{22}} \rho$\\
$\sigma_{12} \sigma_{21} = \sigma_{11} \sigma_{22} \rho^{2}$\\

$\frac{\delta w_{2}}{\delta \mu_{2}} = \frac{1}{\lambda} \frac{1}{\sigma_{11}\sigma_{22}\sigma_{11} - \sigma_{11} \sigma_{22} \sigma_{11} \rho^{2}}$\\
$\frac{\delta w_{2}}{\delta \mu_{2}} = \frac{1}{\lambda} \frac{1}{\sigma_{11}^{2}\sigma_{22} (1 - \rho^{2})}$\\

We find that in each derivative expression, as $\rho$ approaches the absolute value of one, it will make the demoninator in this expression closer to zero, thus the overall expression will greatly increase. Thus we can see that a small change in $\mu$ will result in a large change in $w_{1}$. Thus we prove the claim that the weights are sensitive to changes in $\mu$ when the correlation has an absolute value close to one.\\


\end{document}
